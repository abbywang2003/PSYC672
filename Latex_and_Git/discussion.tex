\section{Discussion}
This study represents one of the first efforts to psychometrically validate a misophonia assessment tool for Mandarin-speaking populations. The findings indicate that the Mandarin version of the DMQ exhibits strong internal consistency and structural validity, positioning it as a promising instrument for both research and clinical applications in China. However, cross-cultural differences in model fit indices suggest that certain aspects of misophonia may manifest differently in Chinese populations compared to Western samples, highlighting the importance of cultural context in understanding this condition.
The current findings largely supported our a priori hypotheses regarding the Mandarin DMQ’s validity. As predicted, strong convergent validity was demonstrated through significant correlations between DMQ subscales and corresponding S-Five dimensions (r = .61-.73), particularly for affective-externalizing and cognitive-threat associations. These robust relationships confirm that both instruments capture overlapping constructs of misophonic reactivity while maintaining sufficient discriminant validity as independent measures. The hypothesized moderate associations with hyperacusis (r = .42-.58) and obsessive-compulsive symptoms (r = .31-.49) were similarly confirmed, supporting the measure’s capacity to identify related yet distinct sensory and psychological processes. Notably, the anticipated weak correlations with religiosity (r = .02-.19) and social desirability (r = -.24-.08) were observed, reinforcing the DMQ’s specificity to misophonia phenomenology rather than general response tendencies or cultural factors.
These validation results position the Mandarin DMQ as meeting contemporary standards for cross-cultural instrument development (Beaton et al., 2000). The measure demonstrated particular strength in assessing core symptom domains (affective, cognitive, physiological), with reliability coefficients (α = .87-.94) exceeding thresholds for clinical use. However, the comparatively weaker performance of coping-related subscales (α = .81-.84) suggests potential cultural modifications may enhance these dimensions’ validity in Chinese contexts. Importantly, the hierarchical regression analyses confirmed the DMQ's incremental validity - its subscales explained 18-22% additional variance in distress outcomes beyond depression and anxiety symptoms alone (all p < .01), fulfilling its intended purpose as a misophonia-specific assessment tool.
Specifically, for convergent validity, the results from the analysis demonstrate strong convergent validity for the DMQ, particularly in its alignment with the S-Five dimensions associated with emotional distress and behavioral responses, such as Threat, Outburst, and Impact (Table 4). All correlations are statistically significant (p < .001), indicating robust relationships between the DMQ components and the S-Five dimensions. Higher correlations generally suggest stronger convergent validity, reinforcing the idea that the DMQ is effectively measuring constructs that are theoretically linked to misophonia.
For discriminant validity, the DMQ successfully differentiates misophonia-related symptoms from general psychological distress and unrelated constructs. It shows strong convergent validity through its high correlations with relevant measures such as depression, anxiety, hyperacusis, and obsessive-compulsive symptoms. At the same time, the DMQ maintains discriminant validity by showing low correlations with constructs such as religiosity and social desirability. This balance confirms that the DMQ is specifically measuring misophonia-related symptoms and coping strategies, supporting its utility as a valid tool for assessing misophonia in both clinical and research settings.


