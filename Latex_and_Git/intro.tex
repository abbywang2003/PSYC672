\section{Introduction}

Misophonia is a condition marked by intense negative reactions—emotional, physiological, and behavioral—to specific everyday sounds, such as chewing, breathing, or environmental noises \citep{Jastreboff2001} . These auditory triggers most commonly include eating-related sounds, nasal and throat sounds, and other repetitive noises, with eating sounds frequently reported as the most distressing (\citep{Vitoratou2021a}). Individuals with misophonia often experience acute emotional responses such as anger and disgust, accompanied by physiological symptoms like increased heart rate, sweating, and muscle tension. These reactions can lead to considerable functional impairment, including avoidance behaviors, social withdrawal, and even aggressive outbursts.
Network analysis enables an understanding of the interconnections among subscales, providing insights into which parts of the measure are most central to the others. This study employed network analysis to examine the interconnections among DMQ subscales and identify the most central components of misophonia symptomatology.
