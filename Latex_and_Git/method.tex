\section{Method}
Network analysis was conducted on DMQ data from 144 adults with varying levels of misophonia symptoms. Four network models were examined: overall misophonia, symptoms, beliefs, and impairment. Sex differences were also explored.
\subsection{Participants}
Participants aged 18 years or older and fluent in Mandarin were eligible for inclusion. Exclusion criteria included self-reported severe learning or intellectual disabilities. An information sheet was provided at the start of the survey, and participants gave informed consent before completing the online questionnaires. This work is reviewed and approved by the Duke Kunshan University Institutional Review Board (Ref: 2024SW125). 
\subsection{Materials and Procedure}
The Duke Misophonia Questionnaire (DMQ) is the first psychometrically validated self-report measure of misophonia developed for an English-speaking population (Rosenthal et al., 2021). The DMQ underwent a rigorous two-phase development process. In Phase 1, items were generated and iteratively refined using input from multiple stakeholders, including misophonia sufferers, their family members, and professional experts. In Phase 2, a large community sample of adults (n = 424) recruited through Amazon’s Mechanical Turk completed DMQ candidate items and other measures for psychometric analyses. 