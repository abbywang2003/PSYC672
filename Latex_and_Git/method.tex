\section{Method}
Network analysis was conducted on DMQ data from 144 adults with varying levels of misophonia symptoms. Four network models were examined: overall misophonia, symptoms, beliefs, and impairment. Sex differences were also explored.
\subsection{Participants}
Participants aged 18 years or older and fluent in Mandarin were eligible for inclusion. Exclusion criteria included self-reported severe learning or intellectual disabilities. An information sheet was provided at the start of the survey, and participants gave informed consent before completing the online questionnaires. This work is reviewed and approved by the Duke Kunshan University Institutional Review Board (Ref: 2024SW125). 
\subsection{Materials and Measures}
\textit{Duke Misophonia Questionnaire (DMQ)}
The Duke Misophonia Questionnaire (DMQ) is the first psychometrically validated self-report measure of misophonia developed for an English-speaking population (Rosenthal et al., 2021). The DMQ underwent a rigorous two-phase development process. In Phase 1, items were generated and iteratively refined using input from multiple stakeholders, including misophonia sufferers, their family members, and professional experts. In Phase 2, a large community sample of adults (n = 424) recruited through Amazon’s Mechanical Turk completed DMQ candidate items and other measures for psychometric analyses. 

\textit{Selective Sound Sensitivity Syndrome Scale (S-Five)} 
The Selective Sound Sensitivity Syndrome Scale (S-Five) consists of two sections: S-Five Experiences and S-Five Trigger Checklist. The S-Five Experiences is a 25-item measure designed to assess the severity of misophonia (Vitoratou et al., 2021b). Each item is rated on an 11-point rating scale, ranging from 0 (not at all true) to 10 (completely true), resulting in a total score ranging from 0 to 250. Vitoratou et al. (2023) established a cut-off of 87 or higher on the total score to indicate significant misophonia. In this study, we used a validated Mandarin version for Chinese populations, which, in its original validation (Vitoratou et al., 2022), showed a replicated five-factor structure and demonstrated satisfactory internal consistency across all factors, with both Cronbach’s α and McDonald’s ω coefficients of 0.88 or higher. The S-Five Trigger Checklist (S-Five-T) was developed to assess the nature and intensity of various trigger sounds (Vitoratou et al., 2021b). Designed for flexibility, the S-Five-T allows researchers to adjust the number of triggers used. In this study, we administered the 37 trigger sounds from the original S-Five validation study, which have been validated in a Chinese population, demonstrating acceptable reliability, convergent validity, and discriminant validity (Vitoratou et al., 2022). The original response options for emotional reactions were also retained: no feeling, irritation, distress, disgust, anger, panic, other feelings (negative), and other feelings (positive). For each trigger item, respondents first select their primary emotional reaction and then rate the intensity of that reaction (henceforth, trigger intensity) on a scale from 0 (does not bother me at all) to 10 (unbearable/causes suffering). 
